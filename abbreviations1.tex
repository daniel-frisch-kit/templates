%%%%%%%%%%%%%%%%%%%%%%%%%%%%%%%%%%%%%%%%%%%%%%%%%%%%%%%%%%%%%%%%%%%%%%%%%%%%%
%%
%% Mathematical Formulae
%%
%%%%%%%%%%%%%%%%%%%%%%%%%%%%%%%%%%%%%%%%%%%%%%%%%%%%%%%%%%%%%%%%%%%%%%%%%%%%%

\def\vec#1{\underline{#1}} % vector
\def\mat#1{{\mathbf #1}} % matrix 
\def\rvec#1{\vec{\pmb{#1}}} % random vector
\def\rv#1{{\pmb{#1}}} % random variable
\def\Cov#1{\op{Cov}\!\braces{#1}} % covariance 
\def\Var#1{\op{Var}\!\braces{#1}} % variance
\def\E#1{\op{E}\!\braces{#1}} % expectation value
\def\EHat#1{\hat{\op{E}}\!\braces{#1}} % expectation value with hat
\def\Menge#1{{\cal #1}} % set font
\def\trace#1{\op{trace}(#1)} % trace spur 
\def\1_2{{\frac{1}{2}}} % 1/2 fraction half
\def\dd{{\,\op{d}}} % "d" operator for integration
\def\T{ ^\top } % Transpose 
\def\op#1{{\operatorfont{#1}}} % operatorfont
\def\Gauss{{\cal N}}

\DeclareMathOperator{\rect}{rect}
\DeclareMathOperator{\I}{i}
\DeclareMathOperator{\real}{real}
\DeclareMathOperator{\sinc}{sinc}
\DeclareMathOperator{\erf}{erf}
\DeclareMathOperator{\xlog}{xlog}
\DeclareMathOperator{\Ei}{Ei}
\DeclareMathOperator{\diag}{diag}
\DeclareMathOperator{\EVOp}{E}
\DeclareMathOperator*{\argmin}{arg\,min}
\DeclareMathOperator*{\argmax}{arg\,max}

\def\ds{\displaystyle}
\def\ns{\textstyle}

\def\NewN{\mathbb{N}} % {{\rm I\hspace{-.17em}N}}
\def\NewR{\mathbb{R}} % {{\rm I\hspace{-.17em}R}}
\def\NewZ{\mathbb{Z}} % {{\sf Z\hspace{-.35em}Z}}
\def\NewB{\mathbb{B}} % {{\rm I\hspace{-.17em}B}}
\def\NewC{\mathbb{C}}


%%%%%%%%%%%%%%%%%%%%%%%%%%%%%%%%%%%%%%%%%%%%%%%%%%%%%%%%%%%%%%%%%%%%%%%%%%%%%
%%
%% Cross references
%%
%%%%%%%%%%%%%%%%%%%%%%%%%%%%%%%%%%%%%%%%%%%%%%%%%%%%%%%%%%%%%%%%%%%%%%%%%%%%%

% Use \cref{label} for everything: figures, equations, sections, ...
% \usepackage{cleveref} % \cref{fig:abc}, \cref{eq:3}
% cleveref adaptions: 
%\creflabelformat{equation}{#2(#1)#3} % #1: number; #2: href begin; #3: href end
\crefname{equation}{}{} % prefix 
\crefname{figure}{Figure}{Figures} % figure prefix 
\crefname{paragraph}{Paragraph}{Paragraphs} % paragraph prefix 
\crefname{chapter}{Chapter}{Chapters} % chapter prefix 
\crefname{section}{Section}{Sections} % section prefix 
\crefname{lemma}{Lemma}{Lemmas} 
\crefname{theorem}{Theorem}{Theorems} 

% Compatibility
\def\Eq#1{\cref{#1}}
\def\Chap#1{\cref{#1}}
\def\Sec#1{\cref{#1}}
\def\SubSec#1{\cref{#1}}
\def\Ex#1{\cref{#1}}
\def\Fig#1{\cref{#1}}
\def\Tab#1{\cref{#1}}
\def\Alg#1{\cref{#1}}
\def\Line#1{\cref{#1}}
\def\Appendix#1{\cref{#1}}
\def\Def#1{\cref{#1}}


%%%%%%%%%%%%%%%%%%%%%%%%%%%%%%%%%%%%%%%%%%%%%%%%%%%%%%%%%%%%%%%%%%%%%%%%%%%%%%
%% Brackets
%%%%%%%%%%%%%%%%%%%%%%%%%%%%%%%%%%%%%%%%%%%%%%%%%%%%%%%%%%%%%%%%%%%%%%%%%%%%%%

\newcommand\bk[2]{\langle #1,\; #2 \rangle}          % \bk{a}{b}     <a,b> (bra ket)
\newcommand\paren[1]{\left( #1 \right)}              % \paren{a}     (a)   (normal parentheses)
\newcommand\brackets[1]{\left[ #1 \right]}           % \brackets{a}  [a]   (square brackets)
\newcommand\braces[1]{\left\lbrace #1 \right\rbrace} % \braces{a}    {a}
\newcommand\abs[1]{\left| #1 \right|}                % \abs{a}       |a|
\newcommand\abss[1]{\left\| #1 \right\|}             % \abss{a}      ||a||
\newcommand\hint[1]{ \quad \left\vert \; #1 \right.} % \hint{x^2}    |hint (on the right of an equation)
\newcommand\ceil[1]{\left \lceil #1 \right \rceil}
\newcommand\floor[1]{\left \lfloor #1 \right \rfloor}


%%%%%%%%%%%%%%%%%%%%%%%%%%%%%%%%%%%%%%%%%%%%%%%%%%%%%%%%%%%%%%%%%%%%%%%%%%%%%%
%% Densities
%%%%%%%%%%%%%%%%%%%%%%%%%%%%%%%%%%%%%%%%%%%%%%%%%%%%%%%%%%%%%%%%%%%%%%%%%%%%%%
\def\fd#1#2{f_{{#1}}^{#2\!}}   % \fd{k}{v}(v) apperas as f_k^v(v) 
\def\fdp#1{\fd{#1}{\op{p}}}    % prior fp
\def\fde#1{\fd{#1}{\op{e}}}    % posterior fe
\def\fdL#1{\fd{#1}{\op{L}}}    % likelihood fL
\def\fdT#1{\fd{#1}{\op{T}}}    % transition density fT



%%%%%%%%%%%%%%%%%%%%%%%%%%%%%%%%%%%%%%%%%%%%%%%%%%%%%%%%%%%%%%%%%%%%%%%%%%%%%%
%% Figures
%%%%%%%%%%%%%%%%%%%%%%%%%%%%%%%%%%%%%%%%%%%%%%%%%%%%%%%%%%%%%%%%%%%%%%%%%%%%%%

% Figure
% Usage: \Figure{width} {label} {filename} {caption}
% Generating Code: abc.m
\newcommand{\Figure}[4]{%
	\begin{figure*}[t]
		\begin{center}
			\includegraphics[width=#1]{figures/#3}
		\end{center}
		\caption{#4}
		\label{#2}
	\end{figure*}
}


% Subfigure. Usage: 
% ----------------------
%\begin{figure*} 
%	\newcommand{\SubFigureWidth}{.3} \centering 
%	\Subfigure{\SubFigureWidth\textwidth} {label} {filename} {caption}
%	\Subfigure{\SubFigureWidth\textwidth} {label} {filename} {caption}
%	\caption{abc} 
%   \label{fig:abc} 
% % Generating Code: abc.m
%\end{figure*}
% ---------------------
\newcommand{\Subfigure}[4] {%
	\subcaptionbox{#4 \label{#2}}
	{\includegraphics[width=#1]{figures/#3}}
}
\begin{comment}
\newcommand{\Subfigure}[4]{%
	\begin{subfigure}[b]{#1} 
		\centering 
		\captionsetup{justification=centering}
		\includegraphics[width=\textwidth] {figures/#3} 
		\caption{#4} 
		\label{#2}
	\end{subfigure} 
}
\end{comment}





